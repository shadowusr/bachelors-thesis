\subsection{Обзор}
{\color{red}
	TODO}
	
%В большинстве случаев мобильные приложения разрабатываются для платформ iOS и Android, причем для каждой из них есть свой нативный (предусмотренный создателем) программный интерфейс. Однако сегодня все чаще встречаются предложения разработки на React Native. Спрос на работу с этим фреймворком мы видим и среди наших клиентов, 90\% из которых делают выбор именно в пользу разработки на React Native.
%
%Что вообще такое React Native?
%
%React Native — это open source фреймворк, который был представлен Facebook в 2015 году. То есть ему еще всего 4 года, и поэтому в сети можно найти немало критики в адрес разработки на RN. Но на сегодняшний день уже можно говорить о зрелости самого фреймворка и билдеров, которые делают возможной кроссплатформенную разработку мобильных приложений.
%
%Как и в уже привычном для всех React, в React Native пользовательский интерфейс создается декларативно, а на JS программируется дерево элементов пользовательского интерфейса. Разница заключается лишь в том, что ReactJS создает элементы UI в браузере, а React Native позволяет преобразовать стандартные элементы VDOM в нативные визуальные элементы для каждой отдельной взятой мобильной платформы. Кстати, сегодня ходят разговоры о расширении React Native за пределы поддержки одних только Android и iOS. Уже сейчас существует возможность разработки под Windows. А в скором времени, вероятно, фреймворк будет поддерживать также macOS и Apple tvOS.
%
%В чем заключаются плюсы React Native?
%
%Преимущество React Native заключается в том, что разработка ведется на базе хорошо известной библиотеки React, но при этом отображение приложений происходит так, как будто они были разработаны нативно для каждой платформы. У разработки на React Native есть сразу несколько плюсов.
%}

%Вопрос автоматического решения текстовых captcha методами машинного обучения уже неоднократно поднимался исследователями в прошлом\cite{prev-research-2, prev-research-3, prev-research-4}. Тем не менее, большинство методов, представленных ранее, либо нацелены на решение captcha строго определенного типа, либо требуют большого набора данных для обучения (на создание которого как правило уходит большое количество времени и ресурсов). С эволюцией и усилением текстовых captcha некоторые предыдущие подходы теряют свою актуальность\cite{prev-research-1}.
%
%В этой работе будут использоваться современные алгоритмы машинного обучения, в числе которых Pix2Pix-сети, SimGAN-сети и CNN-сети, для создания универсального метода решения captcha, не требующего большого вмешательства человека при обучении моделей.
%
%Так, проблема нехватки данных для обучения может быть устранена путем автоматического создания большого количества картинок при помощи генератора с заведомо известным ответом для каждого созданного изображения captcha и последующего улучшения изображений при помощи генеративно-состязательной сети для повышения схожести с реальными данными. При использовании этого метода возможно обучить генеративную сеть созданию реалистичных текстовых captcha из искусственно сгенерированных изображений на небольшом наборе настоящих изображений, в данном случае, около 1 тысячи.

Процесс разработки приложения состоит из нескольких шагов:
\begin{enumerate}[1)]
	\item Постановка задач и проектирование архитектуры.
	
	\item Разработка пользовательского интерфейса.
	
	\item Разработка базовых переиспользуемых компонентов.
	
	\item Реализация основного функционала приложения с использованием результатов предыдущих шагов.
\end{enumerate}

Далее эти этапы будут рассмотрены подробнее.