\subsection{Сбор данных и создание датасета настоящих данных}
На этом начальном этапе нам нужно просто получить около тысячи изображений captha и решений к ним. В качестве целевого сайта будем рассматривать vk.com. Для появления captcha достаточно «спровоцировать» механизмы защиты выполнением однотипных действий. 

При возникновении необходимости решения captcha обратимся к стороннему сервису, работники которого выполнят решение вручную, после чего произведем повторный запрос к vk API для того, чтобы убедиться в правильности решения. Если решение оказалось корректным, сохраним изображение и соответвующий ему ответ, после чего начнем процесс с начала.

Таким образом, построение первоначального набора данных состоит из таких этапов:
\begin{enumerate}[1)]
	\item выбор подходящего сообщества и содержимого постов;
	\item попытка размещения поста;
	\item получение изображения captcha, обращение к стороннему сервису для получения решения;
	\item изображение с captcha и решение сохраняется для дальнейшего использования.

\end{enumerate}

Этот процесс автоматизируем при помощи небольшого скрипта на языке PHP. В листинге \ref{lst:captcha-gatherer} приводится его немного упрощенная версия.

\begin{lstlisting}[language=PHP,basicstyle=\fontsize{11}{11}\selectfont,tabsize=4,breaklines=true,caption={Скрипт для получения и сохранения решений captcha.},captionpos=b,label={lst:captcha-gatherer}]
$client = new GuzzleHttp\Client([
	'base_uri' => 'https://api.vk.com/method/',
]);
$anticaptcha = new Anticaptcha\ImageToText();

$sentPostsCount = 0;
while ($sentPostsCount < 1000) {
	$requestParams = [
		'form_params' => [
			'access_token' => $config->vkToken,
			'v' => '5.95',
			'owner_id' => $config->postContent,
			'message' => $config->groupId,
		],
	];
	
	$response = $client->request('POST', 'wall.post', $requestParams);
	$response = json_decode($response->getBody()->getContents(), true);
	
	if (@$response['error']['error_code'] != 14) {
		continue; // skip the iteration if captcha is not needed
	}
	
	$captchaImage = file_get_contents($response['error']['captcha_img']); // loading a captcha image
	$requestParams['captcha_sid'] = $response['error']['captcha_sid'];
	
	$anticaptcha->setFile($captchaImage);
	$anticaptcha->createTask();
	$anticaptcha->waitForResult();
	
	$requestParams['captcha_key'] = $anticaptcha->getTaskSolution(); // getting solution
	
	// second request to check whether captcha was solved right
	$response = $client->request('POST', 'wall.post', $requestParams);
	$response = json_decode($response->getBody()->getContents(), true);
	
	if (isset($response['error'])) {
		continue; // solution was incorrect
	}
	
	// saving the image in case it was solved correctly
	saveImage($captchaImage, $requestParams['captcha_key']);
	$sentPostsCount++;
}
\end{lstlisting}