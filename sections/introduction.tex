\section*{Введение}
\addcontentsline{toc}{section}{\protect\numberline{}Введение}
В настоящее время изучение иностранных языков широко распространено. Существует большое количество учебников, курсов и приложений, посвященных изучению дополнительных языков. Изучение и запоминание новых слов и выработка правильного произношения -- одни из основных вопросов, возникающих перед любым человеком, осваивающим иностранный язык.

Автоматизация изучения языков имеет положительные стороны как для ученика, так и для учителя. Программные средства по обучению иностранным языкам позволяют поддерживать постоянный контакт с учеником и непрерывно отслеживать ошибки, не требуя участия учителя. Преподавателю же такие инструменты позволяют сократить время на выставление оценок и рутинную работу с заданиями.

При разработке программного обеспечения с такой направленностью нужно решить ряд проблем, чтобы использование системы было эффективным. Так, средство проверки произношения должно корректно обрабатывать сильно искаженную речь не очень хорошо владеющего языком человека и предоставлять подробный разбор ошибок, по которому пользователь сможет работать над улучшением. Дополнительно к разбору ошибок желательно предоставлять эталон произношения в аудио формате, сгенерированный для изучаемого слова или фразы. Для обеспечения понимания лексического значения, система должна подбирать примеры использования, чтобы пользователь мог увидеть контекст употребления слова.

Таким образом можно выделить ряд задач, который должна решать система по автоматизации обучения иностранным языкам:
\begin{enumerate}
	\item Планомерное обогащение словарного запаса;
	\item Регулярные проверки прогресса;
	\item Проверка произношения и анализ ошибок;
	\item Генерация аудио с эталонным произношением для заданного слова или фразы;
	\item Генерация предложений с примерами использования слова или фразы;
	\item Учёт интересов пользователя по темам при выборе изучаемых слов;
	\item Подбор сложности изучаемого материала по уровню знаний студента;
\end{enumerate}
В данной работе рассматриваются методы решения каждой из задач с использованием алгоритмов машинного обучения, а также практическая реализация в виде кросс-платформенного приложения на React Native.

Актуальность работы обусловлена популярностью обучения иностранным языкам и развитием технических средств, в частности, в сфере IT-технологий, способных повысить продуктивность занятий. Изучение по крайней мере одного иностранного языка является обязательным в большинстве образовательных учреждений. Происходит появление новых подходов в образовательном процессе с использованием возможностей в сферах ML и пользовательских интерфейсов. Инструменты, позволяющие систематически работать над языковой фонетикой, расширением словарного запаса и грамматикой в формате мобильного приложения с элементами алгоритмов машинного обучения, направлены на сокращение временных затрат пользователя и улучшение качества подачи материала.

Новизна работы заключается в использовании алгоритмов машинного обучения для автоматической проверки и отработки произношения иностранной речи. Такой подход к обучению позволит пользователю осознанно улучшать навыки произношения иностранных фонем, слов и фраз, без вмешательства человека. В данной работе представлена конкретная реализация этой идеи в формате мобильного приложения, сочетающая в себе преимущества предыдущих подходов с удобством использования мобильного приложения конечными потребителями.

Помимо работы над произношением приложение позволяет пользователю постепенно обогащать свой словарный запас, при этом не расходуя дополнительного времени. Обучение строится на методике периодического повторения и закрепления материала путем тестирования.

Работа состоит из трех разделов: \hyperref[sec:section-1]{1) Алгоритмы машинного обучения в изучении иностранных языков}, \hyperref[sec:section-2]{2) Разработка кросс-платформенного приложения на React Native} и \hyperref[sec:section-3]{3) Архитектура и используемые технические средства}. В первом разделе представлена теоретическая основа и обзор алгоритмов машинного обучения, которые применяются для решения поставленных задач. Второй раздел содержит информацию о практической реализации приложения на React Native, которое использует результаты первого раздела. Наконец, в третьем разделе, рассматриваются выбранные архитектурные решения и технические детали, касающиеся разработки сервиса.

%В последние годы глубокие нейронные сети (включая рекуррентные) показали превосходство на различных соревнованиях и бенчмарках в сферах обработки естественного языка\cite{nlp-trends}, симуляции голоса \cite{gan-voice-synthesis}, компьютерного зрения\cite{cnn-imagenet} и генерирования реалистичных данных, обладающих заданными признаками. Это связано со многими факторами, в числе которых увеличение доступных вычислительных мощностей, появление новых, более эффективных методов и доступность больших объемов данных для обучения и тестирования нейронных сетей.
%
%Темой данной работы является глубокое обучение, в частности, ге\-не\-ра\-тивно-состязательные нейронные сети, сверточные нейронные сети и Pix2Pix сети. В работе рассматривается практическое применение этих инструментов на примере задачи автоматического прохождения CAPTCHA\footnote{Completely Automated Public Turing test to tell Computers and Humans Apart, далее будет использоваться вариант в нижнем регистре.} тестов.
%
%Актуальность работы обусловлена новизной вышеперечисленных методов, высокими результатами, которые они показывают, а также крайне большой популярностью текстовых captcha для разделения людей от компьютерных программ. Так, крупнейшие сервисы, включая Yandex, VK, Mail.ru, Google и другие, используют текстовые captcha для ограничения деятельности автоматизированных программ.
%
%Цель работы -- детально рассмотреть новейшие методы глубокого обучения и их применимость к проблеме автоматического распознавания текстовых captcha.
%
%Работа состоит из двух разделов: \hyperref[sec:section-1]{Обзор моделей глубокого обучения} и \hyperref[sec:section-2]{Распознавание текстовых captcha с использованием алгоритмов глубокого обучения}. В первом разделе будет представлена теоретическая основа и обзор используемых методов, а во втором – их практическое применение для решения проблемы распознавания изображений captcha.
