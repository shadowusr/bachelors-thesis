\subsection{Вывод по разделу}
В разделе представлены исчерпывающие теоретические сведения об алгоритмах машинного обучения, задействованных в решении задач мобильного приложения по изучению иностранных языков. Кратко направления работы можно охарактеризовать так:
\begin{itemize}
	\item \textbf{Автоматическая проверка произношения}. Этот класс алгоритмов используется в приложении для предоставления возможности пользователям тренировать и проверять свою устную речь.
	\item \textbf{Синтез речи}. Генерирование речи в формате звукового сигнала используется нами для создания эталонных аудио фрагментов, которые пользователь бы мог прослушать и использовать при обучении.
	\item \textbf{Генерация текста}. Синтез предложений по заданной теме мы применяем для генерирования базы примеров использования слов, которая позволит пользователю наглядно увидеть применение новой лексики в контексте.
	\item \textbf{Классификация текста}. Задача классификации слов и фрагментов текста возникает из-за необходимости разделять контент для пользователей по уровням сложности и сферам интересов.
\end{itemize}

Мы убедились, что все поставленные задачи можно решить с помощью алгоритмов машинного обучения. В рамках раздела был рассмотрен широкий спектр технологий, включая скрытые марковские модели, RNN--сети, CNN--сети, LSTM, Connectionist Temporal Classification, механизмы внимания, GAN--сети, наивный байесовский классификатор и алгоритмы на решающих деревьях.

Практическая ценность информации этого раздела будет продемонстрирована далее, на этапе реализации мобильного приложения, направленного на автоматизацию и повышение эффективности процесса обучения иностранным языкам.
