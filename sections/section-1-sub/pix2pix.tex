\subsection{Pix2Pix сети}
Pix2Pix сети являются одним из видов генеративно-состязательных сетей. Они направлены, прежде всего, на конвертирование одного изображения в другое\cite{pix2pix-overview}, например, превращение черно-белых фотографий в цветные. Pix2Pix сети основаны на условной генеративно-состязательной сети, где целевое изображение генерируется, а исходное изображение является условием\cite[с.~455]{jason-brownlee-gan}. В данном случае, Pix2Pix сеть минимизирует функцию потерь так, чтобы созданное изображение было правдоподобно относительно исходной категории данных и вместе с тем правдоподобно в качестве трансформации исходного изображения.

Упомянутая выше условная генеративно-состязательная сеть -- это расширение обычных GAN сетей, в котором помимо случайного шума на вход подается еще условие, позволяющее контролировать итоговый вывод модели\cite{gan-conditional}. Например, можно управлять классом, которому должен принадлежать создаваемый образец. Так, в Pix2Pix сетях условием является исходное изображение. В последующих абзацах будут рассмотрены детали архитектуры Pix2Pix сетей.