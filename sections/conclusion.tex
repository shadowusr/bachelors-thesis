\section*{Заключение}
\addcontentsline{toc}{section}{\protect\numberline{}Заключение}
В рамках выпускной квалификационной работы были получены следующие результаты:

% \begin{enumerate}[1)]
\begin{enumerate}
	\item  Проведено исследование, анализ и сравнение алгоритмов машинного обучения в контексте изучения иностранных языков:
	\begin{itemize}
		\item Распознавание речи и ошибок произношения с помощью скрытых марковских моделей, CNN, RNN и end--to--end подходов;
		\item Классификация текста с помощью наивного байесовского классификатора, деревьев решений и алгоритмов глубокого обучения;
		\item Генерация текста с использованием рекуррентных нейронных сетей, LSTM и GAN сетей;
		\item Синтез речи с использованием модели WaveNet и архитектуры Tacotron 2;
	\end{itemize}
	\item Реализовано кросс-платформенное мобильное приложение на React Native:
	\begin{itemize}
		\item Разработка макетов пользовательского интерфейса в Sketch;
		\item Реализация приложения с использованием TypeScript, React Native, Firebase SDK;
		\item Настройка инфраструктура разработки: VCS, ESLint, Prettier, конфигурация CI/CD пайплайнов в bitbucket;
	\end{itemize}
	\item Реализован backend для мобильного приложения:
	\begin{itemize}
		\item Реализация аутентификации, взаимодействия с базой данных и облачным хранилищем на Firebase;
		\item Интеграция с платформой облачных вычислений Microsoft Azure;
		\item Интеграция со сторонними API TwinWord и InferKit.
	\end{itemize}
\end{enumerate}

В работе продемонстрированы эффективность и возможности алгоритмов машинного обучения в сфере изучения иностранных языков. Автоматизация образовательного процесса с помощью мобильного приложения и средств ML позволяет пользователю постепенно обогащать свой словарный запас, при этом уделяя внимание фонетике и контексту использования слов.

Результаты работы показывают, что современные алгоритмы машинного обучения способны решать задачи распознавания и синтеза речи на уровне, достаточном для использования в сфере образования. Технологии React Native, а также платформы Firebase и Microsoft Azure позволяют быстро создавать гибкие и масштабируемые программные решения.

В качестве направлений для будущей работы можно определить дальнейшее повышение качества генерации текста, распознавания речи и разработку новых методов обучения, таких, как, например, синтез речи в реальном времени для реализации устного диалога с пользователем. Это требует доработки существующих подходов, а возможно --- создания совершенно новых решений.
